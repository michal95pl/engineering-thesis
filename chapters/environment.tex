\section*{Środowisko}
Przedstawione rozwiązanie zostało zaimplementowane w języku Python. Podyktowane to było przede wszystkim dużą ilością dostępnych wysokopoziomowych bibliotek, które znacznie przyśpieszają opracowanie i testowanie algorytmów.

Ze względu na bardzo duży koszt obliczeniowy trenowania sieci, został wykorzystany serwer z procesorem Intel Xeon E5-2686v4 wraz z 128GB pamięci RAM oraz kartami graficznymi Nvidia RTX 3080 12gb oraz NVIDIA Tesla v100 16gb. Jako system operacyjny został użyty PROXMOX na którym została uruchomiona maszyna wirtualna z systemem Ubuntu. Jako miejsce do przechowywania danych treningowych i testowych zostały wykorzystane dwa dyski SSD NVME o pojemności 1TB każdy, spięte w macierz RAID 0. Pozwoliło na uzyskanie akceptowalnej prędkości odczytu i zapisu, a także odpowiedniej pojemności.

Taka konfiguracja sprzętowa pozowoliła na efektywne trenowanie i testowanie dwóch modeli równolegle, gdzie przy ograniczonym czasie było to niezbędne.

\hspace{0.5cm}

\begin{figure}[h]
\centering
\includegraphics[width=1\textwidth]{images/proxmox.png}
\caption{Interfejs zarządzania serwerem}
\end{figure}

\newpage

\section*{Wykorzystane biblioteki}
Język Python, który jest interpretowanym językiem wysokiego poziomu. W celu zapewnienia akceptowalnej wydajności, wymagane jest używanie bibliotek napisanych w językach niższego poziomu.
W celu implementacji przedstawionego rozwiązania zostały wykorzystane między innymi:

\begin{itemize}
	\item \textbf{NumPy}: służy do obliczeń naukowych. Umożliwia bardzo szybkie operacje na macierzach i wektorach.
	\item \textbf{Matplotlib}: umożliwia tworzenie wykresów i wizualizację danych.
	\item \textbf{PyTorch}: biblioteka do implementacji sieci neuronowych. Umożliwia elastyczne tworzenie architektur sieci oraz łatwe wykorzystanie GPU.
\end{itemize}
