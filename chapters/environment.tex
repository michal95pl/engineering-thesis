\section*{Środowisko}
\addcontentsline{toc}{chapter}{Środowisko}
Przedstawione rozwiązanie zostało zaimplementowane w języku Python. Podyktowane to było przede wszystkim dużą liczbą dostępnych wysokopoziomowych bibliotek, które w znacznym stopniu przyśpieszają opracowanie i testowanie algorytmów.

Ze względu na duży koszt obliczeniowy przeprowadzanych eksperymentów, został skonfigurowany dedykowany serwer dostosowany do obsługi wieloprocesowych narzędzi. Został on wyposażony w procesor Intel Xeon E5-2686v4 wraz z 128GB pamięci RAM. Do trenowania modeli wykorzystano dwie karty graficzne: Nvidia RTX 3080 12gb oraz NVIDIA Tesla v100 16gb. W celu przechowywania danych treningowych i testowych zostały użyte dwa dyski SSD NVME o pojemności 1TB każdy, połączone w macierz RAID 0. Umożliwiło to na uzyskanie odpowiedniej prędkości odczytu i zapisu. Całość funkcjonowała pod kontrolą systemu Proxmox, który umożliwia łatwe zarządzanie zasobami sprzętowymi poprzez interfejs webowy.
Powyższa konfiguracja sprzętowa pozwoliła na efektywne trenowanie i testowanie dwóch modeli równolegle, co było kluczowe dla realizacji eksperymentów w założonym czasie.

\hspace{0.5cm}

\begin{figure}[h]
\centering
\includegraphics[width=1\textwidth]{images/proxmox.png}
\caption{Interfejs zarządzania serwerem}
\end{figure}

% \newpage

% \section*{Najważniejsze biblioteki}
% Język Python jest językiem interpretowanym. Aby zapewnić wydajność niezbędną w obliczeniach numerycznych, konieczne jest wykorzystanie bibliotek, których kluczowe moduły zostały zaimplementowane w wydajnych językach niższego poziomu. W procesie implementacji rozwiązania wykorzystano między innymi następujące pakiety:

% \begin{itemize}
% 	\item \textbf{NumPy}: podstawowa biblioteka do obliczeń numerycznych. Umożliwia efektywną pracę na wektorach i macierzach.
% 	\item \textbf{Matplotlib}: umożliwia wizualizację danych w formie wykresów.
% 	\item \textbf{PyTorch}: biblioteka do implementacji sieci neuronowych. Umożliwia elastyczne tworzenie architektur sieci oraz łatwe wykorzystanie GPU.
% 	\item \textbf{Chess}: dedykowana biblioteka do obsługi gry w szachy. Umożliwia reprezentację planszy oraz tworzenie i walidację ruchów.
% \end{itemize}
