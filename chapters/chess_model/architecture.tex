\setcounter{chapter}{4}
\chapter[Architektura aplikacji serwerowej {[\textit{Michał Lichtarski}]}]{Architektura aplikacji serwerowej}

W tej części zostanie przybliżona architektura aplikacji serwerowej oraz protokół komunikacji na której opiera się cała praca.

\section{Komunikacja}
Do komunikacji między dwoma częściami przedstawianej pracy, została stworzona klasa \textit{Communication}. Działa ona w oparciu o \textit{sockety}, gdzie każdy klient jest obsługiwany w osobnym wątku. Ze względu na ograniczenia wydajnościowe, niezależnie od ilości klientów działa tylko jedna instancja silnika szachowego. W momencie otrzymania zapytania od klientów, wiadomości są kolejkowane i obsługiwane w kolejności ich nadejścia.

Podobnie to wygląda po stronie klienta, gdzie klasa \textit{ClientCommunication} odpowiada za wysyłanie oraz odbieranie wiadomości od serwera. Podobnie jak w przypadku serwera działa ona w osobnym wątku oraz korzysta z kolejkowania wiadomości.

Wiadomości są w postaci \textit{JSON} i zawierają następujące 2 pola: \textit{command} oraz \textit{boards}. Pole \textit{command} zawiera komendę do wykonania. W obecnej wersji jest tylko jedna komenda \textit{get\_move}, która przekazuje do silnika szachowego aktualne stany plansz i zwraca odpowiedź z najlepszym ruchem. Pole \textit{boards} zawiera listę stanów planszy w formacie \textit{FEN}. Silnik podejmuje decyzję o ruchu na podstawie do 3 stanów. Po ich analizie zwraca odpowiedź z polem \textit{move}, które zawiera najlepszy ruch.

\section{Interfejs aplikacji serwerowej}
Aplikacja serwerowa obługiwana jest poprzez prosty interfejs konsolowy, pozwalający na wywoływanie funkcji. Po wpisaniu komendy \textit{help} wyświetlana jest lista dostępnych komend wraz z ich krótkim opisem. Dodatkowo wyświetlone są parametry jakie obsługują. Kwadratowe nawiasy oznaczają parametry opcjonalne. W celu lepszej automatyzacji serwera, w pliku konfiguracyjnym pod nazwą \textit{startCommands} można wpisać listę komend wraz z argumentami, które zostaną automatycznie wywołane po starcie aplikacji.

Komendy są pogrupowane w trzy kategorie:
\begin{itemize}
    \item \textbf{Ogólne - } Pozwalają na tworzenie serwera oraz wypisanie statusu aplikacji serwerowej.
    \item \textbf{Zarządzenie siecią neuronową - } Pozwalają na ładowanie oraz tworzenie modeli, a także na konwertowanie zbioru danych.
    \item \textbf{Ewaluacja - } Pozwalają na ewaluacje całego silnika szachowego lub samej sieci neuronowej.
\end{itemize}

\newpage

\begin{figure}[h]
\centering
\includegraphics[width=1\textwidth]{images/app_interfejs.png}
\caption{Interfejs aplikacji serwerowej}
\end{figure}