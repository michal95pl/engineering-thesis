\section*{Wstęp}

Żyjemy w dynamicznie rozwijającym się technologicznie świecie, w którym algorytmy uczenia maszynowego odgrywają coraz to większą rolę w różnych dziedzinach życia. Możemy je spotkać nawet w szachach. Prawie każdy z nas zna strony internetowe gdzie można rywalizować z różnymi graczami, czy nawet z modelami. Dla zobrazowania jak algorytmy szachowe są skomplikowanym zadaniem Claude Shannon w swoim artykule z 1950 roku \textit{Programming a Computer for Playing Chess} oszacował, że liczba wszystkich różnych kombinacji partii w szachach wynosi około $10^{120}$, co jest nazwane dzisiaj liczbą Shannon'a. Również opisał jak powinien wyglądać algorytm heurystyczny, który bierze pod uwagę najbardziej obiecujące ruchy "Select the variations to be explored by some process so that the machine does not waste its time in totally pointless variations.". Gdyż jak sam podkreślił, metodą brute force jest to nieosiągalne. Dało to początek rozwoju algorytmów takich jak minimax, czy monte Carlo tree search, które są obecnie szeroko stosowane do dzisiaj.

Jak bardzo zmieniły się algorytmy szachowe na przestrzeni lat dobrze obrazuje książka Huberta Dreyfusa \textit{Alchemy and Artificial Intelligence} z 1965 roku, w której autor napisał że nie ma obecnie algorytmu szachowego, który byłby wstanie grać na amatorskim poziomie "Still no chess program can play even amateur chess". Gdzie zaledwie 20 lat później, w 1988 roku, program szachowy \textit{Deep Thought} wygrał z arcymistrzem szachowym Bentem Larsenem. W 1997 roku komputer \textit{Deep Blue} pokonał Kasparowa w meczu z standardową kontrolą czasu, a w 2017 roku program AlphaZero opracowany przez Google pokonał algorytmy szachowe, takie jak Stockfish i Elmo.