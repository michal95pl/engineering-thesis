\section*{Streszczenie}

Celem pracy jest zastosowanie algorytmu drzewiastego typu monte carlo wraz z siecią neuronową do wykonywania ruchów w szachach. Do wstępnego trenowania sieci będzie wykorzystany zbiór danych gier wybitnych szachistów. W celu doskonalenia gry, będzie użyty algorytm alpha zero, gdzie sieć będzie sama generować zbiór danych na podstawie którego będzie się douczać. Poniższa praca przedstawi działanie powyższych algorytmów oraz bibliotek, które zostały użyte. Dodatkowo w pracy będą zamieszczone problemy, które pojawiły się podczas implementacji.

\textbf{Słowa kluczowe} – monte carlo tree search, sieć neuronowa, alpha zero, szachy
