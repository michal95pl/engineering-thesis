\section*{Biblioteka chess}
Biblioteka chess dostarcza zestaw podstawowych metod do pracy z szachami w pythonie. Umożliwia między innymi odczyt pliku pgn, stworzenie obiektu reprezentującego planszę szachową i wykonywanie na niej ruchów wraz z sprawdzaniem ich poprawności, a nawet umożliwia implementacje gotowego algorytmu jak stockfish. Działanie biblioteki opiera się na dwóch częściach: wizualizacyjnej oraz logicznej (todo: może zmienić).

\section*{Warstwa wizualizacyjna}
Warstwa wizualizacyjna jest bardzo prosta, gdyż przechowuje informacje o planszy w postaci ascii. Cała plansza jest zapisywana w postaci 8 linii oddzielonych znakiem "/". Każdy znak reprezentuje pierwszą literę angielskiej nazwy figury np: p od pawn, b od bishop itp. Dodatkowo w przypadku białych figur, jest to wielka litera, a w przypadku czarnych mała. Cyfry oznaczają ilość pustych pól. Na przykład: PPPP1PP oznacza cztery białe pionki, jedno puste pole oraz dwa białe pionki. Po zakodowanych polach planszy, występują dodatkowe informacje odzielonych spacją: kolor który wykonuje ruch, prawa do roszady, pola na których można wykonać bicie w przelocie, licznik półruchów oraz numer rundy. Konstruktor obiektu planszy przyjmuje między innymi jako argument \textit{fen}, czyli właśnie przedstawiony format. Daje to możliwość kopiowania oraz bardzo łatwego przesyłania dużej ilości informacji o planszy, co zostanie w dalszej części pracy wykorzystane.

\begin{lstlisting}[
    language=Python,
    caption=Przykład warstwy wizualizacyjnej,
    inputencoding=utf8,
    basicstyle=\ttfamily\footnotesize,
    backgroundcolor=\color{gray!10},
    frame=single,
    showspaces=false,
    showstringspaces=false,
    numbers=none
]
>>> board = chess.Board("r1bqkb1r/pppp1Qpp/2n2n2/4p3/2B1P3/8/PPPP1PPP/RNB1K1NR b KQkq - 0 4")
>>> print(board)
r . b q k b . r
p p p p . Q p p
. . n . . n . .
. . . . p . . .
. . B . P . . .
. . . . . . . .
P P P P . P P P
R N B . K . N R
\end{lstlisting}

\section*{Warstwa logiczna}
Warstwa logiczna jest znacznie bardziej rozbudowana. W postaci kilku liczb 64 bitowych przechowuje informacje o planszy. Każdy bit odpowiada jednemu polu na planszy. W ten sposób są zapisywane następujące informacje:

\begin{itemize}
    \item rodzaj figury
    \item kolor figury
    \item czy pole jest puste
    \item czy pole jest atakowane przez białe figury
    \item czy pole jest atakowane przez czarne figury
    \item czy figura jest z promocji
    \item czy figura ma prawo do roszady
\end{itemize}

\vspace{0.5cm}

\begin{lstlisting}[
    language=Python,
    caption=Przykładowy bitboard warstwy logicznej,
    inputencoding=utf8,
    basicstyle=\ttfamily\footnotesize,
    backgroundcolor=\color{gray!10},
    frame=single,
    showspaces=false,
    showstringspaces=false,
    numbers=none
]
>>> print(board)
r n b q k b n r
p p p p p p p p
. . . . . . . .
. . . . . . . .
. . . . . . . .
. . . . . . . .
P P P P P P P P
R N B Q K B N R

>>> BoardPlus.show_bitboard(board.rooks)
1 0 0 0 0 0 0 1
0 0 0 0 0 0 0 0
0 0 0 0 0 0 0 0
0 0 0 0 0 0 0 0
0 0 0 0 0 0 0 0
0 0 0 0 0 0 0 0
0 0 0 0 0 0 0 0
1 0 0 0 0 0 0 1
\end{lstlisting}

