\section*{Podsumowanie}
\addcontentsline{toc}{chapter}{Podsumowanie}
Celem pracy było zaprojektowanie oraz implementacja kompletnego systemu do gry w szachy, który łączy równoległy silnik szachowy oparty na algorytmie Monte Carlo Tree Search z modułem analizy obrazu fizycznej szachownicy.

Pierwszy element systemu stanowił silnik szachowy wykorzystujący algorytm MCTS wspomagany głęboką siecią neuronową. Jest ona oparta o architekturę rezydualną rozszerzoną o bloki Squeeze and Excitation oraz funkcję aktywacji SiLU. Zbiór danych treningowych pochodzi z Lichess Elite Database, zawierający partie mistrzowskie. 

Kluczowym aspektem pracy było zrównoleglenie algorytmu MCTS zarówno z perspektywy CPU jaki i GPU. Wykorzystano w tym celu metodę wieloprocesowego Tree Parallelization, technikę WU-UCT oraz mechanizm batchowania danych, co pozwoliło na efektywniejsze wykorzystanie zasobów sprzętowych.

Przeprowadzone eksperymenty wykazały wysoką skuteczność głowicy policy sieci neuronowej, jednakże głowica value osiągnęła słabsze wyniki.

Drugim filarem pracy był system identyfikacji figur szachowych na fizycznej planszy. Została ona specjalnie zaprojektowana z kolorowymi znacznikami w narożnikach, a także stworzono dedykowane figury ułatwiające ich detekcję.

Do detekcji planszy i figur szachowych wykorzystano techniki przetwarzania obrazu oraz model sieci neuronowej z warstwami konwolucyjnymi. System został stworzony tak, aby mógł działać w pogorszonych warunkach oświetleniowych z uwzględnieniem cieni.


Oba moduły zostały zintegorwane poprzez interfejs komunikacyjny, tworząc kompletny system do gry w szachy na fizycznej planszy.


Elementami pracy, które mogłyby zostać udoskonalone w przyszłości to przede wszystkim głowica value sieci neuronowej, która osiągnęła słabsze wyniki niż oczekiwano. 

Kolejnym aspektem mogłoby być rozwinięcie modułu analizy obrazu, tak aby działał z wykorzystaniem standardowych figur szachowych i planszy szachowej, bez konieczności stosowania specjalnych znaczników.