\section*{Środowisko}
Przedstawione rozwiązanie zostało zaimplementowane w języku Python. Podyktowane to było przede wszystkim dużą ilością dostępnych wysokopoziomowych bibliotek, które znacznie przyśpieszają opracowanie i testowanie algorytmów.

Za względu na duży koszt obliczeniowwy przeprowadznych eksperymentów, został zbudowany dedykowany serwer. Opisywana praca skupia się na implementacji wieloprocesowych narzędzi. Z tego powodu serwer został wyposażony w procesor Intel Xeon E5-2686v4 z 128GB pamięci RAM. Do trenowania modeli zostały wykorzystane dwie karty graficzne: Nvidia RTX 3080 12gb oraz NVIDIA Tesla v100 16gb. Jako miejsce do przechowywania danych treningowych i testowych zostały użyte dwa dyski SSD NVME o pojemności 1TB każdy spięte w macierz RAID 0. Pozwoliło to na uzyskanie odpowiedniej prędkości odczytu i zapisu. Całość funkcjonowała pod kontrolą systemu Proxmox opartego na dystrybucji Debian, który umożliwia łatwe zarządzanie zasobami sprzętowymi poprzez interfejs webowy.
Powyższa konfiguracja sprzętowa pozowoliła na efektywne trenowanie i testowanie dwóch modeli równolegle, co było kluczowe dla realizacji eksperymentów w założonym czasie.

\hspace{0.5cm}

\begin{figure}[h]
\centering
\includegraphics[width=1\textwidth]{images/proxmox.png}
\caption{Interfejs zarządzania serwerem}
\end{figure}

\newpage

\section*{Najważniejsze biblioteki}
Język Python jest językiem interpretowanym. Aby zapewnić wydajność niezbędną w obliczeniach numerycznych, konieczne jest wykorzystanie bibliotek, których kluczowe moduły zostały zaimplementowane w wydajnych językach niższego poziomu. W procesie implementacji rozwiązania wykorzystano między innymi następujące pakiety:

\begin{itemize}
	\item \textbf{NumPy}: podstawowa biblioteka do obliczeń numerycznych. Umożliwia efektywną pracę na wektorach i macierzach.
	\item \textbf{Matplotlib}: umożliwia wizualizację danych w formie wykresów.
	\item \textbf{PyTorch}: biblioteka do implementacji sieci neuronowych. Umożliwia elastyczne tworzenie architektur sieci oraz łatwe wykorzystanie GPU.
	\item \textbf{Chess}: dedykowana biblioteka do obsługi gry w szachy. Umożliwia reprezentację planszy oraz tworzenie i walidację ruchów.
\end{itemize}
