\setcounter{chapter}{5}
\chapter[Przetwarzanie obrazu szachownicy {[\textit{Marcin Ziółkowski}]}]{Przetwarzanie obrazu szachownicy}
 
Proces przygotowania szachownicy do analizy składa się z kilku etapów, mających na celu wyeliminowanie zniekształceń perspektywicznych oraz precyzyjne wykadrowanie obszaru gry. 

\section{Detekcja punktów referencyjnych}

Algorytm rozpoczyna pracę od segmentacji kolorów w przestrzeni barw HSV. Wykorzystywane są cztery markery: 
jeden czerwony oraz trzy zielone, umieszczone w narożnikach fizycznej planszy. 

\begin{itemize}
    \item \textbf{Maskowanie kolorów:} Dla każdego koloru tworzona jest maska binarna przy użyciu progowania adaptacyjnego 
    z uwzględnieniem kalibracji użytkownika. Proces ten ilustrują pliki \texttt{chapters/chess\_identification/images/steps/step1\_green\_mask.png} 
    oraz \texttt{chapters/chess\_identification/images/steps/step1\_red\_mask.png}, które wyodrębniają tylko obszary o zadanej barwie.
    \item \textbf{Morfologia i centroidy:} Na maski nakładana jest operacja otwarcia (\texttt{MORPH\_OPEN}) z kernelem $5 \times 5$ w 
    celu usunięcia szumów. Następnie, za pomocą momentów obrazu (\texttt{cv2.moments}), wyliczane są środki ciężkości wykrytych konturów. 
    Punkty te są wizualizowane jako kolorowe okręgi na obrazie \texttt{chapters/chess\_identification/images/steps/step2\_detected\_centers.png}.
\end{itemize}

\section{Orientacja i transformacja perspektywiczna}

Kluczowym elementem jest poprawna identyfikacja narożników na podstawie wzajemnych odległości punktów zielonych od punktu 
czerwonego, który służy jako punkt odniesienia (\textit{top\_left}).

\begin{itemize}
    \item \textbf{Identyfikacja narożników:} Punkt zielony znajdujący się najdalej od czerwonego klasyfikowany jest jako prawy dolny róg. 
    Poprawność tej logiki weryfikuje obraz \texttt{chapters/chess\_identification/images/steps/step3\_board\_with\_diagonals.png}, przedstawiający przekątne łączące 
    zidentyfikowane narożniki.
    \item \textbf{Rektyfikacja:} Po ustaleniu współrzędnych, obliczana jest macierz transformacji, która rzutuje zniekształcony 
    czworokąt na kwadrat o wymiarach $800 \times 800$ pikseli. Efekt prostowania perspektywy widoczny jest na 
    zdjęciu \texttt{chapters/chess\_identification/images/steps/step4\_perspective\_warped.png}.
\end{itemize}


\section{Precyzyjne kadrowanie i finalna normalizacja}

Ostatnim etapem jest usunięcie zbędnych marginesów poza polem gry. Wykorzystywana jest do tego funkcja \texttt{findChessboardCornersSB}, 
która wyszukuje wewnętrzne przecięcia linii szachownicy.

\begin{enumerate}
    \item \textbf{Subpixel Accuracy:} Pozycje narożników są uściślane z dokładnością podpikselową za pomocą 
    algorytmu \texttt{cornerSubPix}.
    \item \textbf{Kadrowanie:} Na podstawie współrzędnych wykrytych pól wyznaczany jest obszar cięcia z 
    marginesem (\textit{padding}).
    \item \textbf{Finalny wynik:} Rezultatem jest obraz \texttt{chapters/chess\_identification/images/steps/step5\_final\_cropped.png}, który przedstawia 
    idealnie wykadrowaną i znormalizowaną szachownicę, gotową do detekcji bierek.
\end{enumerate}