\section{Komunikacja z serwerem obliczeniowym}

W celu zapewnienia wysokiej wydajności analizy ruchów oraz odciążenia interfejsu użytkownika, 
moduł komunikacji został zaimplementowany jako oddzielny wątek działający w tle, dziedziczący po klasie \texttt{Thread}. 
System wykorzystuje protokół \texttt{TCP/IP} do nawiązywania stabilnego połączenia z serwerem.

\subsection{Protokół przesyłania danych}

Komunikacja opiera się na wymianie komunikatów w formacie \texttt{JSON}, co pozwala na łatwą serializację złożonych struktur danych, 
takich jak stany szachownicy. Proces wymiany informacji przebiega w następujący sposób:

\begin{enumerate}
    \item \textbf{Inicjalizacja:} Klient łączy się z serwerem pod wskazany adres \texttt{host} i numer \texttt{portu}, 
    tworząc strumień danych \texttt{SOCK\_STREAM}.
    \item \textbf{Wysłanie stanu gry:} Do serwera wysyłana jest tablica zawierająca trzy ostatnie stany szachownicy 
    zapisane w notacji \texttt{FEN} (Forsyth-Edwards Notation). Przesłanie sekwencji stanów pozwala serwerowi na 
    lepszą weryfikację dynamiki zmian na szachownicy.
    \item \textbf{Odbiór decyzji:} Serwer po przeanalizowaniu obrazów i stanów przesyła odpowiedź zawierającą informację o 
    wykonanym ruchu.
\end{enumerate}

\subsection{Zarządzanie asynchroniczne}

Zastosowanie kolejki zadań (\texttt{Queue}) pozwala na bezpieczne przesyłanie komunikatów między wątkiem komunikacyjnym a głównym 
wątkiem aplikacji (\texttt{VideoFrameThread}). Dzięki temu interfejs graficzny pozostaje responsywny, podczas gdy dane są 
odbierane w pętli \texttt{while True} i buforowane do momentu ich przetworzenia.

\begin{itemize}
    \item \textbf{Metoda \texttt{send}:} Koduje słownik z danymi (w tym tablicę stanów FEN) do formatu UTF-8 i przesyła go przez gniazdo.
    \item \textbf{Metoda \texttt{run}:} Nieustannie nasłuchuje na przychodzące pakiety danych o rozmiarze do 4096 bajtów.
    \item \textbf{Metoda \texttt{get\_message}:} Pozwala na pobranie zdekodowanego komunikatu o ruchu pionka z kolejki, o ile 
    jest on dostępny.
\end{itemize}