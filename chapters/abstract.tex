\section*{Streszczenie}

Celem niniejszej pracy jest przedstawienie działania oraz implementacji równoległego silnika szachowego opartego na algorytmie przeszukiwania drzewa Monte Carlo wspomaganego siecią neuronową. Zostanie przedstawiony proces przetwarzania danych szachowych wraz z niezbędną modyfikacją biblioteki szachowej.
W pracy zostaną omówione ulepszenia zarówno sieci neuronowej jak i samego MCTS, które zostały wprowadzane na przestrzeni ostatnich kilku lat. Na końcu zostaną przedstawione wyniki ewaluacji samej sieci neuronowej oraz porównanie opisywanego silnika z silnikiem Stockfish.

\textbf{Słowa kluczowe} – monte carlo tree search, sieć neuronowa, szachy, wieloprocesowość
