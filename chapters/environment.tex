\section*{Środowisko}
Przedstawione rozwiązanie zostało zaimplementowane w języku Python. Podyktowane to było przede wszystkim dużą ilością dostępnych wysokopoziomowych bibliotek, które znacznie przyśpieszają opracowanie i testowanie algorytmów.

Ze względu na bardzo duży koszt obliczeniowy trenowania sieci, został wykorzystany serwer z procesorem Intel Xeon E5-2686v4 wraz z 128GB pamięci RAM oraz kartą graficzną NVIDIA RTX 3080 z 12GB pamięci. Jako system operacyjny został użyty PROXMOX na którym zostały uruchomione odpowiednie maszyny wirtualne z systemem Ubuntu desktop z bezpośrednim dostępem do GPU. Takie rozwiązanie pozwala na dużą elastyczność ze względu na łatwy dostęp do zasobów serwera poprzez przeglądarkę internetową z dowolnego miejsca. Ponadto umożliwia nieprzerwane trenowanie modeli.

\section*{Wykorzystane biblioteki}
Język Python, który jest interpretowanym językiem wysokiego poziomu, wymaga używania bibliotek napisanych w językach niższego poziomu, w celu zapewnienia akceptowalnej wydajności.
W celu implementacji przedstawionego rozwiązania zostały wykorzystane następujące biblioteki:

\begin{itemize}
	\item \textbf{NumPy}: służy do obliczeń naukowych; umożliwia bardzo szybkie operacje na macierzach i wektorach.
	\item \textbf{Matplotlib}: umożliwia tworzenie wykresów i wizualizację danych.
	\item \textbf{PyTorch}: biblioteka do implementacji sieci neuronowych; umożliwia elastyczne tworzenie architektur sieci oraz łatwe wykorzystanie GPU, co znacząco przyspiesza trenowanie modeli.
\end{itemize}
