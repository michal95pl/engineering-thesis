\setcounter{chapter}{8}
\chapter[Modele bierek i przygotowanie danych uczących {[\textit{Marcin Ziółkowski}]}]{Modele bierek i przygotowanie danych uczących}

Kluczowym elementem sukcesu klasyfikacji obrazu jest wysoka jakość oraz reprezentatywność zbioru danych treningowych. W niniejszym rozdziale opisano autorskie podejście do fizycznego modelowania bierek oraz proces budowy bazy danych obrazowych wykorzystanych do trenowania sieci splotowej \texttt{ChessCNN}.

\section{Specyfikacja fizycznych tokenów bierek}
W projekcie zdecydowano się na odejście od klasycznych, trójwymiarowych figur szachowych na rzecz autorskich tokenów w formie płaskich krążków (piktogramów). Wybór ten podyktowany był chęcią eliminacji problemu okluzji (zasłaniania pól przez wysokie figury) oraz uproszczeniem procesu ekstrakcji cech.



\subsection{Konstrukcja wizualna i kolorystyka}
Tokeny zostały zaprojektowane jako dwukolorowe układy o wysokim kontraście, wydrukowane w technologii druku 3D. Każdy krążek posiada charakterystyczną strukturę warstwową:
\begin{itemize}
    \item \textbf{Symbol bierki:} Centralnie umieszczony piktogram (np. wieża, pion) w kolorze przypisanym do danej strony (biały lub czarny).
    \item \textbf{Tło piktogramu:} Wypełnienie wnętrza krążka kolorem kontrastowym względem symbolu (np. czarne tło dla białego symbolu).
    \item \textbf{Pierścień zewnętrzny:} Obwódka o kolorze identycznym z kolorem symbolu.
\end{itemize}

Takie podejście rozwiązuje problem "zlewania się" bierki z polem szachownicy. Dzięki zastosowaniu obwódki i tła o przeciwnych barwach, algorytm \texttt{EllipseCrop} jest w stanie precyzyjnie wyznaczyć krawędzie elipsy i wycentrować obiekt nawet w sytuacji, gdy jasność pola szachownicy jest identyczna z kolorem samej figury (np. czarny token na czarnym polu).

\section{Akwizycja danych treningowych}
Zbiór danych został wygenerowany w sposób półautomatyczny przy użyciu skryptu \texttt{setup\_learning\_pieces.py}. Jako bazę wykorzystano zdjęcia rzeczywistej planszy z rozstawionymi tokenami, wykonane w różnych warunkach oświetleniowych.

\subsection{Segmentacja i ekstrakcja}
Proces budowy bazy danych przebiegał według następującego algorytmu:
\begin{enumerate}
    \item Wykrycie siatki szachownicy i transformacja perspektywiczna całego obrazu.
    \item Podział obrazu na 64 pola i przypisanie im etykiet na podstawie wzorcowego rozstawienia bierek (metoda \texttt{seperate}).
    \item Zastosowanie potoku \texttt{EllipseCrop} na każdym polu w celu wycięcia piktogramu z tła i znormalizowania go do rozmiaru $30 \times 30$ pikseli.
\end{enumerate}

\section{Augmentacja danych (Data Augmentation)}
Z uwagi na to, że sieć neuronowa musi być odporna na niedokładne ułożenie tokenów przez gracza (przesunięcia, rotacje), zastosowano agresywną metodę augmentacji syntetycznej w funkcji \texttt{duplicate}.

\begin{lstlisting}[style=codeListingStyle, caption={Fragment skryptu generującego warianty treningowe poprzez rotację}]
def duplicate(square, filename, a, b, steps_r=24, steps_d=12):
    center = (w // 2, h // 2)
    for x in range(steps_r):
        M = cv2.getRotationMatrix2D(center, x*(360/steps_r), 1.0)
        rotated = cv2.warpAffine(square, M, (w, h))
        cropp(rotated, os.path.join(filename, f"square_{a}_{b}_{x}.png"))
\end{lstlisting}

Dla każdego wyciętego pola wygenerowano 24 warianty obrócone o kąt co 15 stopni. Dzięki temu model uczy się cech niezmienniczych względem rotacji, co jest kluczowe w systemie, gdzie kamera może być zamontowana pod dowolnym kątem względem graczy.

\section{Charakterystyka zbioru danych}
Finalny zbiór danych składa się z 6992 próbek podzielonych na 6 klas odpowiadających typom bierek. Podział danych na zbiór treningowy i walidacyjny został dokonany w stosunku 80/20. Dzięki zastosowaniu znormalizowanych tokenów oraz procedury \texttt{EllipseCrop}, uzyskano bardzo wysoką czystość danych wejściowych (ang. \textit{data purity}), co pozwoliło modelowi \texttt{ChessCNN} osiągnąć wysoką celność klasyfikacji już po kilkunastu epokach uczenia.

\begin{figure}[H]
\centering
\includegraphics[width=0.2\textwidth]{chapters/chess_identification/images/Train_material/B4_0_2.png}
\caption{Wieża nr. 1}
\end{figure}

\begin{figure}[H]
    \centering
    \begin{minipage}{0.2\textwidth}
        \centering
        \includegraphics[width=\textwidth]{chapters/chess_identification/images/Train_material/Board4_0_2.png}
        \caption{Wieża nr. 1.1}
    \end{minipage}
    \hfill
    \begin{minipage}{0.2\textwidth}
        \centering
        \includegraphics[width=\textwidth]{chapters/chess_identification/images/Train_material/Board4_0_5.png}
        \caption{Wieża nr. 2.1}
    \end{minipage}
    \hfill
    \begin{minipage}{0.2\textwidth}
        \centering
        \includegraphics[width=\textwidth]{chapters/chess_identification/images/Train_material/Board4_7_2.png}
        \caption{Wieża nr. 3.1}
    \end{minipage}
    \hfill
    \begin{minipage}{0.2\textwidth}
        \centering
        \includegraphics[width=\textwidth]{chapters/chess_identification/images/Train_material/Board4_7_5.png}
        \caption{Wieża nr. 4.1}
    \end{minipage}
\end{figure}