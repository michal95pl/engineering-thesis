\setcounter{chapter}{5}
\chapter[Kadrowanie obrazu szachownicy {[\textit{Marcin Ziółkowski}]}]{Przetwarzanie obrazu szachownicy}
 
Proces przygotowania szachownicy do analizy składa się z kilku etapów, mających na celu wyeliminowanie zniekształceń perspektywicznych oraz precyzyjne wykadrowanie obszaru rozgrywki. W poniższym rozdziale zostaną przedstawione poszczególne kroki tego procesu.

\section{Detekcja punktów referencyjnych} 

Algorytm rozpoczyna pracę od segmentacji kolorów w przestrzeni barw HSV. Wykorzystywane są cztery markery: 
jeden czerwony oraz trzy zielone, umieszczone w narożnikach fizycznej planszy.

\begin{itemize}
    \item \textbf{Maskowanie kolorów:} Dla każdego koloru tworzona jest maska binarna przy użyciu progowania adaptacyjnego z uwzględnieniem kalibracji użytkownika. Proces ten ilustruje rysunek 6.1, gdzie są wyodrębniane obszary o zadanej barwie.
    \item \textbf{Morfologia i centroidy:} Na maski nakładana jest operacja otwarcia (\texttt{MORPH\_OPEN}) z kernelem $5 \times 5$ w 
    celu usunięcia szumów. Następnie, za pomocą momentów obrazu (\texttt{cv2.moments}), wyliczane są środki ciężkości wykrytych konturów. 
    Punkty te są wizualizowane jako kolorowe okręgi na obrazie, co przedstawia rysunek 6.2.
\end{itemize}

\begin{figure}[h]
\centering
\includegraphics[width=0.4\textwidth]{chapters/chess_identification/images/steps/step1_red_mask.png}
\hspace{0.5cm}
\includegraphics[width=0.4\textwidth]{chapters/chess_identification/images/steps/step1_green_mask.png}
\caption{Maski kolorów czerwonego i zielonego}
\end{figure}

\begin{figure}[h]
\centering
\includegraphics[width=0.6\textwidth]{chapters/chess_identification/images/steps/step2_detected_centers.png}
\caption{Wykryte centroidy}
\end{figure}

\newpage

\section{Orientacja i transformacja perspektywiczna}

Następnym kluczowym elementem jest poprawna identyfikacja narożników na podstawie wzajemnych odległości punktów zielonych od punktu 
czerwonego, który służy jako punkt odniesienia (\textit{top\_left}).

\begin{itemize}
    \item \textbf{Identyfikacja narożników:} Punkt zielony znajdujący się najdalej od czerwonego klasyfikowany jest jako prawy dolny róg. 
    Poprawność tej logiki weryfikuje lewy obraz rysunku 6.3, przedstawiający przekątne łączące 
    zidentyfikowane narożniki.
    \item \textbf{Rektyfikacja:} Po ustaleniu współrzędnych, obliczana jest macierz transformacji, która rzutuje zniekształcony 
    czworokąt na kwadrat o wymiarach $800 \times 800$ pikseli. Efekt prostowania perspektywy widoczny jest na prawym obrazie rysunku 6.3.
\end{itemize}

\begin{figure}[h]
\centering
\includegraphics[width=0.4\textwidth]{chapters/chess_identification/images/steps/step3_board_with_diagonals.png}
\hspace{0.5cm}
\includegraphics[width=0.4\textwidth]{chapters/chess_identification/images/steps/step4_perspective_warped.png}
\caption{Identyfikacja narożników oraz efekt transformacji perspektywicznej}
\end{figure}

\newpage

\section{Precyzyjne kadrowanie i finalna normalizacja}

Ostatnim etapem jest usunięcie zbędnych marginesów poza polem gry. Wykorzystywana jest do tego funkcja \texttt{findChessboardCornersSB}, 
która wyszukuje wewnętrzne przecięcia linii szachownicy.

\begin{enumerate}
    \item \textbf{Subpixel Accuracy:} Pozycje narożników są uściślane z dokładnością podpikselową za pomocą 
    algorytmu \texttt{cornerSubPix}.
    \item \textbf{Kadrowanie:} Na podstawie współrzędnych wykrytych pól wyznaczany jest obszar cięcia z 
    marginesem (\textit{padding}).
    \item \textbf{Finalny wynik:} Rezultatem jest obraz, który przedstawia idealnie wykadrowaną i znormalizowaną szachownicę, gotową do detekcji bierek. Co ilustruje rysunek 6.4.
\end{enumerate}

\begin{figure}[h]
\centering
\includegraphics[width=0.4\textwidth]{chapters/chess_identification/images/steps/step5_final_cropped.png}
\caption{Finalny wynik po kadrowaniu i normalizacji szachownicy}
\end{figure}